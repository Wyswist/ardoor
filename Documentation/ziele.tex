\documentclass[main.tex]{subfiles}
\begin{document}

\section{Ziele}

\subsection{Plattform-Unabhängigkeit}
Die Anwendung soll auf möglichst vielen verschiedenen Platformen betrieben werden können. Die vorerst gewählten Zielplattformen sind Android, iOS, Mac OS X und Linux. Um dies bewerkstelligen zu können, muss ein so grosser Teil des Codes wie möglich Plattformunabhängig in C++ geschrieben werden. Auf den einzelnen Platformen sollen dann nur noch das GUI und der Zugriff auf die Hardwarekomponenten spezifisch gehandhabt werden. Weiterhin soll dies auch die Basis für den weiteren Verlauf des Projektes während der Bachelortesis legen.

\subsection{Positionsbestimmung}
Der wichtigste Teil unserer Arbeit ist es, eine Möglichkeit zu finden die Position der Kamera gegenüber der Projektionsfläche möglichst korrekt zu bestimmen. Hierzu soll auf die von OpenCV gegebenen Mittel zurückgegriffen werden. Das Ziel ist es, die Applikation soweit zu bringen, dass es möglich ist ein 3D-Objekt korrekt auf ein Schachbrett zu projizieren. Das Rendering des Objektes soll mit Hilfe von OpenGL ES umgesetzt werden. Dabei ist die Performance noch zweitrangig. Hier ebenfalls, soll die Grundlage für die Bachelor Thesis erarbeitet werden, wo dann die Anbauten direkt in das Livebild eingerechnet werden sollen.

\subsection{Theorie}
Neben dem Verständnis von OpenCV soll ebenfalls die Theorie hinter den eingesetzten Mechanismen erarbeitet werden. Um zu einem späteren Zeitpunkt Optimierungsmöglichkeiten für die Applikation zu finden, ist ein grundlegendes Verständnis der Algorithmen welche für die Positionsbestimmung eingesetzt werden nötig.

\end{document}