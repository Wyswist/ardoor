\documentclass[main.tex]{subfiles}
\begin{document}

\section{Ausblick}

In der nun bevorstehenden Bachelor-Thesis geht es nun als erstes darum, den Fehler in der perspektivischen Projektion zu beheben. Dies ist der Grundstein für die weitere Arbeit. In einem weiteren Schritt muss das Marker-basierte Konzept abgelöst werden. Wollen wir Türen automatisch erkennen, so müssen wir ohne jegliche Marker auskommen. Diese zwei Punkte sollen vorerst ebenfalls in der Desktop App umgesetzt werden.
\paragraph{}
Basierend auf einer funktionierenden Desktop Anwendung, können dann die mobilen Applikationen entwickelt werden. Der Hauptteil der Arbeit wird hier darin bestehen, Optimierungsmöglichkeiten zu finden um die Framerate zu erhöhen. Mögliche Mittel sind hier OpenCL aber auch NEON. NEON ist zwar eine ARM-Spezifische Technologie, da aber sowohl alle gängigen Android Devices als auch alle iPhone ab der vierten Generation auf ARMv7 und höher aufbauen, ist NEON auch in einem gewissen Mass plattformunabhängig.
\paragraph{}
Da wir gegen Schluss unseres Projekts hauptsächlich auf dem Desktop entwickelt haben und wir möglichst schnell etwas visuell darstellen wollten, haben wir die grafische Darstellung vor allem mit alten OpenGL-Funktionen umgsetzt. Die moderne Art der OpenGL-Programmierung basiert vor allem auf direkter Programmierung der GPU via Shader und verwendet das \textit{OpenGL Core Profile}. Da auf Smartphones OpenGL ES 2 verwendet wird, wollen wir während der Bachelor-Thesis ebenfalls unsere Codebasis auf diese Version migrieren, so dass wir auf sämtlichen Plattformen denselben OpenGL-Code wiederverwenden können.
\paragraph{}
Als letztes Ziel haben wir eine Saubere und gut dokumentierte Codebasis. Auch wenn wir unsere Ziele eventuell nicht zu 100\% erreichen, soll die Arbeit als Grundlage für weitere Forschungen verwendet werden können. Hierzu muss die Hürde für die Übernahme unserer Arbeit so tief wie möglich gehalten werden.

\end{document}
