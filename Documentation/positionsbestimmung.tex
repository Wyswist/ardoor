\documentclass[main.tex]{subfiles}
\begin{document}

\section{Positionsbestimmung}
Die Positionsbestimmung besteht aus drei elementaren Teilen. Der Kamerakalibrierung, welche das Bild entzerrt und wichtige Parameter der Kamera zurückliefert, welche für die Bestimmung der Perspektive wichtig sind. Anhand dieser Daten wird mittels der von OpenCV bereitgestellten Funktion die Perspektive von der Kamera zur Projektionsfläche bestimmt. Zuletzt wird mit Hilfe von Rodrigues die Rotation der Projektionsfläche angepasst. Diese drei Teile wollen wir hier nun etwas detaillierter erklären.

\subsection{Kamerakalibrierung}


\subsection{Perspektive}
solvePnP beschreiben

\subsection{Rotation}
Rodrigues beschreiben

\end{document}